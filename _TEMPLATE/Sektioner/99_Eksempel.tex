%*******************************************************************************
% Eksempel
%*******************************************************************************
\section{Eksempel}
Denne fil viser hvordan diverse opsætning af et LaTeX-dokument foregår. Der vises bl.a. hvordan figurer og tabeller indsættes, ligesom der også vises, hvordan kodestykker indsættes.\\
\smallskip\\
Se afsnit \ref{subsec:setup_new_document} for guide til hvordan et nyt dokument sættes op.

%===============================================================================
% Opsætning
%===============================================================================
\subsection{Opsætning af nyt dokument}
\label{subsec:setup_new_document}
Følg nedenstående steps ved opsætning af et nyt dokument:
\begin{enumerate}
	\item Kopier hele ``\_TEMPLATE'' mappen og indsæt den det ønskede sted.
	\item Omdøb mappen.
	\item Omdøb filen ``TEMPLATE.tex''.
	\item Ændr ``\textbackslash title'' i den netop omdøbte fil.
	\item Opdater forfatter og dato i filen ``00\_VersionHistory''.
	\item Indsæt den (eller de) ønskede sektioner i mappen ``Sektioner'', og gå i gang.
\end{enumerate}

%===============================================================================
% Figurer og tabeller
%===============================================================================
\subsection{Figurer og tabeller}
Ved indsætning af figurer og tabeller kan der medgives forskellige optioner, der fortæller programmet, hvor figuren eller tabellen ønskes indsat. Optionen medgives ved at indsætte den i firkantede paranteser, efter det \textit{begin}-statement der starter figurens eller tabellens sektion. Eksempler på brug kan ses ved at kigge i .tex-filen for dette dokument.\\
\medskip\\
\textbf{De optioner der kan medgives er følgende:}
\begin{itemize}
	\item h (here) - Placer hvor den står i tekst.
	\item t (top) - Toppen af siden.
	\item b (bottom) - Bunden af siden.
	\item p (page) - På sin egen side.
	\item H (override) - Vigtigt at den står lige her.
\end{itemize}

\newpage
%-------------------------------------------------------------------------------
% Figurer
%-------------------------------------------------------------------------------
\subsubsection{Figurer}
Figur \ref{fig:eksempel_test} viser et test billede.
\begin{figure}[H]
	\center
	\includegraphics[width=0.85\textwidth,height=0.85\textheight,keepaspectratio]{test.png}
	\caption{Caption Text}
	\label{fig:eksempel_test}
\end{figure}

%-------------------------------------------------------------------------------
% Tabeller
%-------------------------------------------------------------------------------
\subsubsection{Tabeller}
Tabel \ref{tab:eksempel_tabel} viser en test tabel.
\begin{table}[H]
	\rowcolors{2}{gray!25}{white}
	\centering
	\caption{Dette er en tabel}
	\begin{tabular}{|l|l|l|}
		\hline
		\rowcolor{gray!50}
		\textbf{Studienummer} & \textbf{Navn}\\ [5px]
		\hline
		201607589 & Jakob Bonde Nielsen\\
		\hline
		201604998 & Jonas Nielsen\\
		\hline
		201607110 & Kasper Juul Hermansen\\
		\hline
		20112806 & Martin Lynge Dalgaard\\
		\hline
		201607436 & Simon Lassen\\
		\hline
		201605114 & Stefanie Ruaya Nielson\\
		\hline
	\end{tabular}
	\label{tab:eksempel_tabel}
\end{table}

\newpage
%===============================================================================
% Kode
%===============================================================================
\subsubsection{Kode}
Herunder kan ses Listing \ref{lst:eksempel_kode}, som er indsat via en fil.

\lstinputlisting[language=c++, caption={Kode Eksempel}, label={lst:eksempel_kode}]{Kode/EksempelKode.cpp}
